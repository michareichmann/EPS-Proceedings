By now the technology of diamond detectors is well established in high energy physics. Many of the experiments are already using \acp{BCM} or \acp{BLM} based on diamonds and their impact is steadily increasing. As one of the first pixel projects the ATLAS \ac{DBM} started taking data and was recommissioned for the \SI{13}{\tera\electronvolt} collisions.\\
The diamond material was proven to be very radiation hard and the signal behaviour after the irradiation with various particle species and energies is very well understood for both \ac{sc}\ac{CVD} and \ac{pCVD} diamonds. In extensive studies it was also found out that \ac{pCVD} diamond detectors work reliably and show no signal dependence up to an incident particle flux of \SI{10}{\mega\hertz\per\centi\meter^2}. This could also be shown for irradiated detectors up to fluence of \SI{5e14}{n_{eq}\per \centi\meter^2}.\\
There is also great progress in the development of even more radiation hard devices. The working principle of both 3D strip and pixel detectors could be proven with great success down to cell sizes of \SI{100x100}{\micro\meter}. For the first time more than \SI{80}{\%} of the created charge in the material could be read out. The efficiency of the column drilling process is now above \SI{99}{\%} and the total efficiency of the 3D pixel detectors is very high with \SI{98.5}{\%}.