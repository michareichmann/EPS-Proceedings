By now the technology of diamond detectors is well established in high energy physics. Many of the experiments are already using \acp{BCM} or \acp{BLM} based on \ac{CVD} diamonds. As one of the first pixel projects the ATLAS \ac{DBM}  was recommissioned for the \SI{13}{\tera\electronvolt} collisions and started taking data.\\
The diamond material was proven to be very radiation tolerant and the signal behaviour after the irradiation with various particle species and energies is well understood for both \sccvd and \pcvd diamonds. In extensive studies it was found that \pcvd diamond detectors work reliably and show no signal dependence up to an incident particle flux of \SI{20}{\mega\hertz\per\centi\meter^2}. This was also shown for irradiated detectors up to fluence of \SI{5e14}{n_{eq}\per \centi\meter^2}.\\
There is also great progress in the development of more radiation tolerant devices. The working principle of both 3D strip and pixel detectors was proven with great success down to cell sizes of \SI{100x100}{\micro\meter}. For the first time more than \SI{80}{\%} of the created charge in the material was read out. The efficiency of the column drilling process is now above \SI{99}{\%} and the relative efficiency of the 3D pixel detectors is \SI{99.3}{\%} compared to a silicon detector.