The upgrade of the \ac{LHC} to the \ac{HL-LHC} from \SIrange{2023}{2025}{} \cite{hllhc} will push the luminosity limits even above the original design values of the \ac{LHC} and will therefore hopefully give us even more insights in the fundamental nature of the universe. In 2028 an instantaneous luminosity of \SI{5e34}{\per\centi\meter\squared\per\second} is aspired. The innermost tracking layer at a distance of \SI{\sim30}{\milli\meter} will encounter pixel hit rates of \SI{3}{\giga\hertz\per\centi\meter^2} in this environment and is expected to be exposed to a total fluence of \SI{2e16}{n_{eq}\per \centi\meter^2} by 2028 \cite{auzinger}. The current pixel detectors are designed to withstand \SI{\sim300}{\per\femto\barn} and thus the full detector would have to be replaced about every year. This fact led to research and development of various radiation hard detector designs and materials.\\
Its large displacement energy of \SI{42}{\electronvolt\per atom} and a high band gap of \SI{5.5}{\electronvolt} make diamond an excellent candidate for such a radiation tolerant detector which is why the RD42 Collaboration is investigating \ac{sc} and \ac{p} \ac{CVD} diamond as an alternative for precision tracking detectors for over two decades. In order to grow high quality detector grade diamonds, RD42 collaborates with industrial companies. All shown results are acquired with \sccvd diamonds produced by Element Six Technologies \cite{e6} and \pcvd diamonds produced by II-VI Incorporated \cite{II6}. The main difference between the two types are their sizes of \SI{\sim0.25}{\centi\meter\squared} for \sccvd and up to \SI{6}{inch} for \pcvd and the \sfrac{2}{3} smaller signal in \pcvd \cite{felix}.
% The two companies use propriety \ac{CVD} processes to fabricate their products. Both diamond types are grown on homo-epitaxial substrates with the difference that for \ac{sc}\ac{CVD} another \ac{sc}\ac{CVD} diamond is used as substrate and thus its size is limited to \SI{\sim0.25}{\centi\meter\squared}. However, for the \ac{p}\ac{CVD} a diamond powder can be used as a substrate whereby it can be grown to wafers of diameters up to \SI{6}{inch} \cite{felix}. 
In various studies it was shown that compared to corresponding silicon detectors, diamond is at minimum three times more radiation hard \cite{deboer}, has at least a two times faster charge collection \cite{pernegger} and its thermal conductivity is four times higher \cite{zhao}.\\
It is essential for all modern collider experiments to have an online monitoring of the beam conditions. Since it is important to have these detectors as close as possible to the beam all of the four main experiments at the \ac{LHC} are using detectors with diamond sensors. ATLAS \cite{gorisek}, ALICE, CMS \cite{bartz} and LHCb \cite{domke} all make use of various \acp{BCM} and/or \acp{BLM} based on both \ac{CVD} type diamonds for live background estimations and luminosity measurements.\\
Due to the very high particle fluxes and radiation doses expected for the \ac{HL-LHC} it is very important to understand the behaviour of future detectors in this environment. The RD42 Collaboration has studied \ac{CVD} diamond detectors with irradiation doses up to \SI{2.2e16}{p\per\centi\meter^2}. In order to build even more radiation hard detectors, a new technology - 3D detectors \cite{3D} - is investigated. The design of these detectors allows to heavily reduce the drift distance of the created charge carriers without reducing the total number of the created electron-hole pairs. Since the the signal behaviour of diamonds at high fluxes is uncertain, high rate studies are performed at \ac{PSI} with nearly \acp{MIP} and tunable particle fluxes from the order of \SI{1}{\kilo\hertz\per cm^2} up to the order of \SI{10}{\mega\hertz\per cm^2}.