The upgrade of the \ac{LHC} to the \ac{HL-LHC} from \SIrange{2023}{2025}{} \cite{hllhc} will push the luminosity limits even above the original design values of the \ac{LHC} and will therefore hopefully give us even more insights in the fundamental nature of the universe. The in 2028 aspired instantaneous luminosity of \SI{5e34}{\per\centi\meter\squared\per\second}  is equivalent to a fluence of \SI{2e16}{n_{eq}\per \centi\meter^2} \cite{auzinger} for the innermost tracking layer at a distance of \SI{\sim30}{\milli\meter} from the interaction point. In this environment pixel hit rates of \SI{3}{\giga\hertz\per\centi\meter^2} are expected. The current pixel detectors are designed to withstand \SI{\sim300}{\per\femto\barn} and thus the full detector would have to be replaced about every semester. This fact lead to research and development of various radiation hard detector designs and materials.\\
Its large displacement energy and the high band gap (\SI{5.5}{\electronvolt} at \SI{305}{\kelvin}) make diamond an excellent candidate for such a radiation tolerant detector which is why the RD42 Collaboration is investigating single-crystal and poly-crystalline \ac{CVD} diamond as an alternative for precision tracking detectors for over two decades. In various studies it was found out that diamond is minimum three times more radiation hard \cite{deboer}, has at least a two times faster charge collection \cite{bla} and its thermally conductivity is four times higher \cite{blub} than corresponding silicon detectors.