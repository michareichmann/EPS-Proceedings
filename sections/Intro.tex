The upgrade of the \ac{LHC} to the \ac{HL-LHC} from \SIrange{2023}{2025}{} \cite{hllhc} will push the luminosity limits even above the original design values of the \ac{LHC} and will therefore hopefully give us more insights in the fundamental nature of the universe. In 2028 an instantaneous luminosity of \SI{5e34}{\per\centi\meter\squared\per\second} is expected. In this environment the innermost tracking layer at a distance of \SI{\sim30}{\milli\meter} to the \ac{IP}
%will encounter pixel hit rates of \SI{3}{\giga\hertz\per\centi\meter^2} 
is expected to be exposed to a total fluence of \SI{2e16}{n_{eq}\per \centi\meter^2} by 2028 \cite{auzinger}. This fluence is equivalent to an integrated luminosity of \SI{\sim3000}{\per\femto\barn}, but since the  current pixel detectors are designed to withstand \SI{\sim300}{\per\femto\barn} the full detector would have to be replaced about every year. This led to research and development of new radiation tolerant detector designs and materials.\par
Its large displacement energy of \SI{42}{\electronvolt\per atom} and a high band gap of \SI{5.5}{\electronvolt} make diamond an excellent candidate for such a radiation tolerant detector which is why the RD42 Collaboration is investigating \ac{sc} and \ac{p} \ac{CVD} diamond as an alternative for precision tracking detectors for over two decades. In order to grow high quality detector grade diamonds, RD42 works together with industrial companies. All results in this paper were acquired with \sccvd diamonds produced by Element Six Technologies \cite{e6} and \pcvd diamonds produced by II-VI Incorporated \cite{II6}. The main difference between the two types of diamonds are their sizes of \SI{\sim0.25}{\centi\meter\squared} for \sccvd and up to \SI{6}{inch} for \pcvd and the smaller signal in \pcvd \cite{felix}.
% The two companies use propriety \ac{CVD} processes to fabricate their products. Both diamond types are grown on homo-epitaxial substrates with the difference that for \ac{sc}\ac{CVD} another \ac{sc}\ac{CVD} diamond is used as substrate and thus its size is limited to \SI{\sim0.25}{\centi\meter\squared}. However, for the \ac{p}\ac{CVD} a diamond powder can be used as a substrate whereby it can be grown to wafers of diameters up to \SI{6}{inch} \cite{felix}. 
In various studies it was shown that compared to corresponding silicon detectors, diamond is at minimum three times more radiation hard \cite{deboer}, has at least a two times faster charge collection \cite{pernegger} and its thermal conductivity is four times higher \cite{zhao}.\par
It is essential for all modern collider experiments to have an online monitoring of the beam conditions as close as possible to the beam \cite{hllhc}. Due to the high radiation in that regime presently all of the four main experiments at the \ac{LHC} are using detectors with diamond sensors. ATLAS \cite{gorisek}, ALICE \cite{hempel}, CMS \cite{bartz} and LHCb \cite{domke} all make use of various \acp{BCM} and/or \acp{BLM} based on both \ac{CVD} type diamonds for live background estimations and luminosity measurements.\par
Due to expected high particle flux and expected radiation dose for the \ac{HL-LHC} it is very important to understand the behaviour of future detectors in this environment. The RD42 Collaboration has studied \ac{CVD} diamond detectors with irradiation doses up to \SI{2.2e16}{p\per\centi\meter^2}. In order to build more radiation tolerant detectors, a new technology - 3D detectors \cite{parker} - in diamond is being investigated \cite{3D} . The 3D design of these detectors heavily reduces the drift distance of the created charge carriers without reducing the total number of the created electron-hole pairs. Since the particle flux of the \ac{HL-LHC} will be in completely new regime, high rate studies are performed at \ac{PSI} with nearly \acp{MIP} and tunable particle fluxes from the order of \SI{1}{\kilo\hertz\per cm^2} up to the order of \SI{10}{\mega\hertz\per cm^2}.