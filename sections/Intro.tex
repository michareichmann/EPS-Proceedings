The upgrade of the \ac{LHC} to the \ac{HL-LHC} from \SIrange{2023}{2025}{} \cite{hllhc} will push the luminosity limits even above the original design values of the \ac{LHC} and will therefore hopefully give us even more insights in the fundamental nature of the universe. The in 2028 aspired instantaneous luminosity of \SI{5e34}{\per\centi\meter\squared\per\second} will be equivalent to a fluence of \SI{2e16}{n_{eq}\per \centi\meter^2} \cite{auzinger} for the innermost tracking layer at a distance of \SI{\sim30}{\milli\meter} from the interaction point. In this environment pixel hit rates of \SI{3}{\giga\hertz\per\centi\meter^2} are expected. The current pixel detectors are designed to withstand \SI{\sim300}{\per\femto\barn} and thus the full detector would have to be replaced about every semester. This fact lead to research and development of various radiation hard detector designs and materials.\\
Its large displacement energy and the high band gap (\SI{5.5}{\electronvolt} at \SI{305}{\kelvin}) make diamond an excellent candidate for such a radiation tolerant detector which is why the RD42 Collaboration is investigating \ac{sc} and \ac{p} \ac{CVD} diamond as an alternative for precision tracking detectors for over two decades. In order to grow high quality detector grade diamonds RD42 collaborates with industrial companies. All shown results are acquired with \ac{sc}\ac{CVD} diamonds produced by Element Six Technologies \cite{e6} and \ac{p}\ac{CVD} diamonds produced by II-VI Incorporated \cite{II6}. The two companies use propriety \ac{CVD} processes to fabricate their products. Both diamond types are grown on homo-epitaxial substrates with the difference that for \ac{sc}\ac{CVD} another \ac{sc}\ac{CVD} diamond is used as substrate and thus its size is limited to \SI{\sim0.25}{\centi\meter\squared}. However for the \ac{p}\ac{CVD} a diamond powder can be used as a substrate by what in can be grown to wafers of diameters up to \SI{6}{inch} \cite{felix}. In various studies it was found out that diamond is minimum three times more radiation hard \cite{deboer}, has at least a two times faster charge collection \cite{pernegger} and its thermal conductivity is four times higher \cite{zhao} than corresponding silicon detectors.\\
Due to the very high particle fluxes and radiation doses expected for the \ac{HL-LHC} it is very important to understand the behaviour of future detectors in this environment. The RD42 Collaboration has studied \ac{CVD} diamond detectors up to irradiation doses of \SI{2.2e16}{p\per\centi\meter^2}. In order to build even more radiation hard detectors a new technology - 3D detectors \cite{3D} - is investigated. The clever design of these detectors allows to heavily reduce the drift distance of the created charge carriers without reducing the total number of the electron-hole pairs. Since the behaviour at high fluxes is uncertain, high rate studies are performed at \ac{PSI} with nearly \ac{MIPs} and tunable particle fluxes from the order of \SI{1}{\kilo\hertz\per cm^2} up to the order of \SI{10}{\mega\hertz\per cm^2} are performed.