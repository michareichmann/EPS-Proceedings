In addition to the radiation studies it is very important to understand the signal behaviour of \ac{CVD} diamonds depending on the incident particle flux
%since the \ac{HL-LHC} will reach values in the low \SI{}{\giga\hertz\per\centi\meter^2} range. 
In order to conduct such a study it is important to be able to vary the particle flux in a big range. The $\uppi$M1 beam line at the \ac{HIPA} at \ac{PSI} can provide beams with continuously tunable fluxes from the order of \SI{1}{\kilo\hertz\per\centi\meter^2} up to \SI{10}{\mega\hertz\per\centi\meter^2} which have a spacing of \SI{19.8}{\nano\second} between each bunch. For these studies a $\uppi^{\z{+}}$ beam with with a momentum of \SI{260}{\mega\electronvolt\per c}  was chosen in order to reach the highest possible flux \cite{pim1}.\par
% \wrapfigcl{Setup.png}{.4}{Rate Setup}{rate1}
%%%%%%%%%%%%%%%%%%%%%%% SETUP %%%%%%%%%%%%%%%%%%%%%%%%%%%
The diamond sensors were measured in a pad geometry and prepared as described in \cite{rainer}. 
% They have a size of approximately \SI{4x4}{\milli\meter} with a chrome-gold metalisation of \SI{3.5x3.5}{\milli\meter} plus a guard ring on front and back side. 
In order to resolve single waveforms at high particle rates the sensors are connected to a fast, low-noise amplifier with a rise time of approximately \SI{5}{\nano\second}. The resulting waveforms are then read out with a DRS4 Evaluation Board at a sampling frequency of \SI{2}{\giga\hertz}. The final diamond pad detectors are then measured in a beam telescope based on the CMS pixel \acp{ROC} PSI46v2 \cite{kornmayer} which provides tracking with an inherent resolution of \SI{\sim70}{\micro\meter}  at the position of the DUT. A better resolution can be achieved by applying a cut on the $\upchi^{2}$ distribution of the tracks. The telescope also provides a trigger of which the area can be masked to increase the efficiency of the data taking. A scintillator is positioned at the end of the telescope to achieve a precise timing of \SI{1}{\nano\second}.\par
%%%%%%%%%%%%%%%%%%%%%%% ANALYSIS %%%%%%%%%%%%%%%%%%%%%%%%
An overlay of \num{30000} resulting waveforms is shown in Figure \vref{rate2}. The most frequent peak at \SI{\sim70}{\nano\second} is caused by the actual particle which was triggered on. The region of \SI{20}{\nano\second} around this mean peak position is called signal region. All the other peaks are from particles of other bunches. Due to the good timing resolution the bunch spacing of the \ac{PSI} beam can be clearly seen in the plot. The bunch just before the signal region is forbidden by the trigger logic and is used to extract the pedestal (base line) of the waveform. The pulse height value is then calculated by the an integral around the maximum value in the signal region.\par
% The optimisation of the integration window was done by choosing the best \ac{SNR}.\\

\figcl{SignalWaveforms30000.png}{.18}{Overlay of 30000 waveforms}{rate2}

In order to check the dependence on the incident particle flux several rate scans with both polarities of the bias voltage and different irradiation doses were performed. The typical scan starts at the minimum flux, goes up to the maximum (up scan) and then goes down to the minimum again (down scan). In addition random scan were done whereby systematic effects were excluded. Figure \vref{rate3} shows the final results for a \pcvd diamond both non-irradiated and irradiated to \SI{5e14}{n_{eq}\per \centi\meter^2}. A pulse height dependence on particle flux of less than \SI{5}{\%} was observed for a flux up to \SI{20}{\mega\hertz\per cm^2}. In addition it can also be seen that there is a slight difference between positive and negative bias which is due to differences in the electronics. After the irradiation the pulse height decreases due to the radiation damage. There was no absolute calibration done yet which is required to relate the pulse height values before and after irradiation.

\figcl{DiaScansII6B21.pdf}{.25}{Pulse height versus incident particle flux for a \pcvd diamond for irradiated and non-irradiated detectors}{rate3}